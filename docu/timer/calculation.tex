\documentclass{article}

\usepackage[a4paper, left=2cm, right=2cm, top=3cm]{geometry}

\usepackage[utf8]{inputenc}
\usepackage[ngerman]{babel}
\usepackage[document]{ragged2e}
\usepackage{listings}

\usepackage{tikz}
\usepackage[siunitx, EFvoltages, straightvoltages]{circuitikz}

\begin{document}

\section*{functions}

\subsection*{blocking}

\_Bool \_tim\_timeout\_blocking(uint16\_t seconds, char scaler)

Die Funktion wartet seconds mal dem scaler.

scaler kann folgende Werte annehmen:

1: 1\newline
m: $10^{-3}$\newline
u: $10^{-6}$

Es kann maximal 65 Sekunden gewartet werden.

Sollte die gewünschte Zeit nicht gewartet werden können wird $0$ (False) retuniert, an sonst $1$ (True).

Wärent gewartet wird ist der Prozessor im Sleepmode, andere interrupts unterbrechen das warten nicht.

\subsection*{nonblocking with callback}

\_Bool \_tim\_timeout\_nonblocking\_with\_callback(uint16\_t seconds, char scaler, void (*callback)())

Die funktion started eine funktion nach einer gegebenen Zeit: seconds mal scaler, das warten blockiert den programmablauf nicht.

scaler kann folgende Werte annehmen:

1: 1\newline
m: $10^{-3}$\newline
u: $10^{-6}$

Es kann maximal 65 Sekunden gewartet werden.

Sollte die gewünschte Zeit nicht gewartet werden können wird $0$ (False) retuniert, an sonst $1$ (True).

\section*{resaults}

\begin{figure}[h]
  \centering
  \includegraphics[scale=0.8]{logic.png}
  
\end{figure}

Es wurde vor den Messpfeilen einmal mit einen non Blocking gestarted und dann eine Blocking. Blocking toggled Channel 0 mit 2Sekunden, und non blocking geht nach 500ms auf high. Dabei sieht man das sich die beiden nicht behindern.

Die Zeit die gewartet wird ist immer etwas länger als die gewünschete weil Schritte wie die Zeit die die ISR braucht um zu starten nicht berücksichtigt wird.
Aber auch zeiten der Konfiguration der Timer werden nicht berücksichtigt.

Beide Funktionen können Gleichzeitig gestarted werden, aber die non blocking muss ausgelöst haben um erneut eine ausführen zu dürfen.


\end{document}