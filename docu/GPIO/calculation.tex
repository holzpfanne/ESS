\documentclass{article}

\usepackage[utf8]{inputenc}
\usepackage[ngerman]{babel}
\usepackage[document]{ragged2e}

\usepackage{tikz}
\usepackage[siunitx, EFvoltages, straightvoltages]{circuitikz}

\begin{document}

\begin{figure}[h]
  \centering
  \begin{circuitikz}
    \ctikzset{multipoles/font={\color{red}\tiny},
      monopoles/vcc/arrow={Triangle[width=0.8*\scaledwidth, length=\scaledwidth]}}
    \draw (0,0) node[qfpchip, num pins=32, external pad fraction=6,rotate=90](C){\rotatebox{-90}{IC1}};

    \node [font=\tiny\ttfamily, left, xshift=-0.5cm] at(C.pin 15) {PB1};
    \node [font=\tiny\ttfamily, right, xshift=0.5cm] at(C.pin 28) {PB5};

    \draw (C.pin 15) -- ++(2cm,0) to[european resistor, l=$R_2$, i=$i_D$] ++(0,-2) to[empty led, l=$\quad D$,v=$U_D$] ++(0,-2) node[ground]{};
    \draw (C.pin 28) -- ++(-1cm,0) to[european resistor, l=$R_1$] ++(0,2) node[vcc]{VCC};
    \draw (C.pin 28) -- ++(-1cm,0) to[push button, l=$S_1$] ++(0,-2) node[ground]{};
  \end{circuitikz}
\end{figure}


Maximaler Eingangsstrom : $I_{in max} = 5mA$
\newline
High Voltage : $VCC = 3.3V$
\newline
\bigskip
$$R_{1 min}=\frac{VCC}{I_{in max}}=\frac{3.3V}{5mA} = 660\Omega$$

Für $R_1$ wurde eine $1M\Omega$ Widerstand gewäht um den Strom so gering wie möglich zu halten.
\newline
\bigskip
LED Strom : $i_D = 20mA$\newline
LED Spannung : $U_D = 1.7V$

$$R_{2 min}=\frac{VCC - U_D}{i_D}=\frac{3.3V - 1.7V}{20mA}=80\Omega$$

Für $R_2$ wurde ein $470\Omega$ Widerstand wurde gewählt damit die LED nicht zu hell ist.
\end{document}